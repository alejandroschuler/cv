\section{\mysidestyle Service and Advocacy}

    \href{https://academic.oup.com/jamia/article/11/1/87/855126}
    {\textbf{Editor}}
    \textsl{JAMIA Student Editorial Board}
    \hfill \textsl{May 2018 to present} \\
    {\small I review several papers per year, moderate the online JAMIA journal club, and participate in the annual editorial board meeting.}

    \href{https://associatedstudents.stanford.edu/gsc}
    {\textbf{Councilor}}
    \textsl{Graduate Student Council}
    \hfill \textsl{May 2017 to May 2018} \\
    {\small The Graduate Student Council (GSC) is an elected 15-member independent entity that apportions a quarter-million dollar yearly budget to support the well-being of all graduate students at Stanford and advocate on their behalf to the provost and other executive administrators. As a councilor, I have worked for graduate student representation on administrative committees, housing rights, and sensible tax policy.}

    \href{http://med.stanford.edu/school/leadership/dean/updates/2015/diversity-and-societal-citizenship.html}
    {\textbf{Invited Member}}
    \textsl{Dean's Task Force on Diversity and Societal Citizenship}
    \hfill \textsl{Jan. 2017 to Jan 2018} \\ 
    {\small I reviewed progress on diversity initiatives and educational opportunities at the Stanford School of Medicine and made recommendations for improvement. Recommendations ensure that trainees are both knowledgeable and active in societal issues that impact health and science, as well as overall diversity, equity and inclusion.}
    
    \textbf{Student Representative}
    \textsl{Stanford Biomedical Informatics Training Program}
    \hfill \textsl{Oct. 2016 to Oct. 2017} \\ 
    {\small I represented student interests by serving on executive committees for program development, curriculum, and admissions for the PhD training program. I reviewed applications and interviewed candidates for the program and contributed to admissions decisions. My participation led to the wider adoption of admissions rubrics and kick-started a review of the curriculum and qualification exam process.} 
    
    \href{https://www.bioaims.com/}
    {\textbf{President}}
    \textsl{Biosciences Association for the Interest of Minority Students}
    \hfill \textsl{June 2016 to July 2017}  \\ 
    {\small The purpose of BioAIMS is to address the needs and concerns of current minority graduate students in the biosciences. As president, I coordinated and empowered the board to execute programming. I also served as a voting member on Stanford School of Medicine faculty committees for admissions and diversity, where I instigated and led the implementation of admissions processes to limit unconscious bias.}

    \textbf{Advocacy Chair} 
    \textsl{Biosciences Association for the Interest of Minority Students}
    \hfill \textsl{June 2015 to June 2016} \\ 
    {\small I worked with the Office of Graduate Education (OGE) to survey graduate students' conceptions of advocacy. The results of this survey were used to inform BioAIMS programming and are still in use today.}
    
    \href{https://stanfordmaleoutreach.weebly.com/}
    {\textbf{Men's Outreach Coordinator}}
    \textsl{Stanford Women's Community Center} 
    \hfill \textsl{Oct. 2014 to Feb. 2017} \\ 
    {\small My goal as men's outreach coordinator was to help men become fulfilled, free, and empathetic by promoting meaningful dialogue and action around gender in the Stanford community. I organized and moderated events, discussions, and media showings. I was awarded a competitive grant to design and execute a four-part event series about diversity within feminist identities and movement tactics, which occurred in early 2016 and included renowned speakers such as Estelle Friedman and Roxane Gay.}