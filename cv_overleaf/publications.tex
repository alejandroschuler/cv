\section{\mysidestyle Publications}

Duan, T., Avati, A., Ding, D. Y., Basu, S., Ng, A. Y.,
\href{https://stanfordmlgroup.github.io/projects/ngboost/}{\textbf{Schuler, A.}}
(2019). 
NGBoost: Natural Gradient Boosting for Probabilistic Prediction. 
\textsl{In Review, AIStats 2019}

Schuemie, M.J., Cepeda, M.S., Suchard, M.A., Yang, J., Tian, Y., \textbf{Schuler, A.}, Ryan, P.B., Madigan, D., Hripcsak, G.
(2019).
Evaluation of methods for population-
level estimation of causal effects in
observational healthcare data
\textsl{In Review, Harvard Data Science Review}

\href{https://arxiv.org/abs/1804.05146}{\textbf{Schuler, A.}}, Gombar, S. Baiocchi, M., Tibshirani, R., Shah, N.
(2018).
A comparison of methods for model selection when estimating individual treatment effects.
\textsl{arXiv preprint} arXiv:1804.05146

Wendling, T., Jung, K., \href{https://onlinelibrary.wiley.com/doi/abs/10.1002/sim.7820}{\textbf{Schuler, A}}., Callahan, A., Shah, N.H., Gallego, B. 
(2018).
Comparing methods for estimation of heterogeneous treatment effects using observational data from healthcare databases. 
\textsl{Statistics in Medicine.}

\href{http://www.jacr.org/article/S1546-1440(17)31668-X/fulltext}{\textbf{Schuler, A.}}, Callahan, A., Jung, K., Shah, N.H. 
(2018). 
Performing an informatics consult: methods and challenges. 
\textsl{Journal of the American College of Radiology}.
	
Vashisht, R. Jung, K., \href{https://www.ncbi.nlm.nih.gov/pmc/articles/PMC5333256/}{\textbf{Schuler, A.}},
Banda, J.M., Callahan, A. Park, R.W. Jin, S. Li, L. Dudley, J.T., Johnson, K., Shervey, M. Xu, H., Wu, Y., Natrajan, K., Hripcsak, G., Jin, P., Zandt, M.V., Reckard, A., Reich, C., Weaver, J., Schuemie, M., Ryan, P. Shah, N.H.  
(2018).
Learning effective treatment pathways from observational data: a multinational retrospective cohort study for Type-2 Diabetes. 
\textsl{JAMA Network Open.}

Wang, J. K., Hom, J., Balasubramanian, S., 
\href{https://linkinghub.elsevier.com/retrieve/pii/S1532046418301795}{\textbf{Schuler, A.}},
Shah, N. H., Goldstein, M. K., ... \& Chen, J. H. (2018). 
An Evaluation of Clinical Order Patterns Machine-Learned from Clinician Cohorts Stratified by Patient Mortality Outcomes. 
\textsl{Journal of Biomedical Informatics.}


\href{https://journals.lww.com/ccmjournal/Abstract/onlinefirst/The_Impact_of_Acute_Organ_Dysfunction_on_Long_Term.96340.aspx}{\textbf{Schuler, A.}}, Wulf, D., Lu, Y., Iwashyna, T.J., Escobar, G.J., Shah, N.H., Liu, V.X. 
(2018).
The impact of acute organ dysfunction on long-term survival in sepsis. 
\textsl{Critical Care Medicine.}

Powers, S., Qian, J., Jung, K., \href{https://arxiv.org/abs/1707.00102}{\textbf{Schuler, A.}}, Shah, N. H., Hastie, T., Tibshirani, R. 
(2017). 
Some methods for heterogeneous treatment effect estimation in high-dimensions. 
\textsl{Statistics in Medicine}.

\href{https://arxiv.org/abs/1711.00083}{\textbf{Schuler, A.}}, Jung, K., Tibshirani, R., Hastie, T., Shah, N.H. 
(2017). 
Synth-validation: selecting the best causal inference method for a given dataset. 
\textsl{arXiv preprint} arXiv:1711.00083.

Adams, J.Y., Rogers, A.J., \href{https://www-atsjournals-org.stanford.idm.oclc.org/doi/pdf/10.1164/ajrccm-conference.2017.195.1_MeetingAbstracts.A5029}{\textbf{Schuler, A.}} Marelich, G.P., Fresco, J.M., Taylor, S.L., Riedl, A.W., Baker, J.M., Escobar, G.J., Liu, V.X. 
(2019).
The association between S/F ratio time-at-risk and hospital mortality in mechanically ventilated patients. 
\textsl{The Permanente Journal}.

Richman, I. B., Fairley, M., Jorgensen, M. E., \href{https://jamanetwork.com/journals/jamacardiology/fullarticle/2551983}{\textbf{Schuler, A.}}, Owens, D. K., \& Goldhaber-Fiebert, J. D. 
(2016). 
Cost-effectiveness of intensive blood pressure management. 
\textsl{JAMA cardiology}.

\href{https://www.ncbi.nlm.nih.gov/pubmed/26776181}{\textbf{Schuler, A.}}, Liu, V., Wan, J., Callahan, A., Udell, M., Stark, D. E., \& Shah, N. H. 
(2016). 
Discovering patient phenotypes using generalized low-rank models. 
\textsl{Pacific Symposium on Biocomputing. Pacific Symposium on Biocomputing} (Vol. 21, p. 144).